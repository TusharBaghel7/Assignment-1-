\documentclass[12pt]{report}

\usepackage{hyperref}
\hypersetup{colorlinks=true,citecolor=blue,
linkcolor=blue,urlcolor=blue}

\title{\underline{ASSIGNMENT 1} \\
Report on medical devices and medical technology }
\author{27 January 2022\\ Roll.no-21111067\\Tushar Baghel\\tbtusharbaghel7@gmail.com\\BIOMEDICAL DEPARTMENT\\NIT RAIPUR\\}
\usepackage{graphicx}
\graphicspath{{images/}}

\begin{document}

\begin{figure}
\centering
\includegraphics[scale=0.5]{nitrr.jpg}
\end{figure}
\maketitle
\clearpage
\tableofcontents
\clearpage

\section{ACKNOWLEDGEMENT}\


I would like to express my special thanks of gratitude to my teacher Dr.Saurab Gupta sir who gave me the golden opportunity to this assignment, which also helped me in doing reasearch and gaining knowledge about medical devices and thier working.\paragraph{}

Secondly,I would also like to thank my parents and friends who helped me in suggesting information for this project.\paragraph{•}


Thanking You\paragraph{•}

Tushar Baghel 

\clearpage

\section{MEDICAL DEVICES}
A medical device is any device intended to be used for medical purposes.A medical device can be any instrument, apparatus, implement, machine, appliance, implant, reagent for in vitro use, software, material or other similar or related article, intended by the manufacturer to be used, alone or in combination for a medical purpose.However, medical devices vary from the simple to the more complex and include anything from medical gloves, thermometers and bedpans through to pacemakers and prosthetics.

\subsection{TYPES}
There are 3 classes of medical devices:
Class I devices are low-risk devices. Examples include bandages, handheld surgical instruments, and nonelectric wheelchairs.
Class II devices are intermediate-risk devices. ...
Class III devices are high-risk devices that are very important to health or sustaining life.

\subsection{Our report on 5 Medical Devices}
In this report we will discuss 5 Medical Devices  based on thier type i.e. Class I , Class II , Class III.\\
5 Medical Devices are -\\
1.Stethoscope\\
2.Chest X-ray\\
3.Ear Thermometer\\
4.Pacemaker\\
5.ultrasonic lithotripter\\
\clearpage

\section{Stethoscope}
The stethoscope is an acoustic medical device for auscultation, or listening to internal sounds of an animal or human body. It typically has a small disc-shaped resonator that is placed against the skin, and one or two tubes connected to two earpieces. A stethoscope can be used to listen to the sounds made by the heart, lungs or intestines, as well as blood flow in arteries and veins. In combination with a manual sphygmomanometer, it is commonly used when measuring blood pressure.

\subsection{But how stethoscope works?}
The stethoscope works on the principle of reflection of sound, the vibration causes a diaphragm to vibrate. The diaphragm vibrates when sound occurs. These high-frequency sounds travel up the hollow plastic tubing into hollow metal earpieces and to the doctor's ears.

The disc and the tube of the stethoscope amplify small sounds such as the sound of a patient's lungs, heart and other sounds inside the body, making them sound louder. The amplified sounds travel up the stethoscope's tube to the earpieces that the doctor listens through.

\subsection{parts and function of stethoscope}
The sounds move through the stethoscope tubing and into the eartubes, where the user can hear the patient's heartbeat, as well as lung and abdominal sounds. A stethoscope has a chestpiece, diaphragm and/or bell, stem, tubing, headset, eartubes, and eartips.
\begin{figure}[h]
\centering
\includegraphics[scale=0.5]{stethoscope}
\caption{Stethoscope}
\end{figure}
\clearpage


\subsection{Types of stethoscope}

 1. Acoustic-Acoustic stethoscopes operate on the  
   transmission of sound from the chest piece, via air-
   filled hollow tubes, to the listener's ears.\\
   
2.Electronic-An electronic stethoscope (or  
   stethophone)   
   overcomes the low sound levels by electronically   
   amplifying body sounds.
   Electronic stethoscopes require the conversion of   
   sound waves to electronic signals which are later 
   amplified and processed to make them optimal for 
   listening.\\

 
3.Recording-Some stethoscopes use a direct audio output 
  that can be used with external recording devices such  
  as a laptop. By doing this, the healthcare 
  professional can compare the current recording with a 
  previously recorded signal and improve diagnostics 
  capabilities.\\


4.Fetal Stethoscope- A fetal stethoscope is an acoustic 
  stethoscope that is used to listen to the sounds of  
  the fetus. The stethoscope that comes in the shape of 
  a trumpet, can diagnose any abnormalities in pregnant 
  women.\\


5.Doppler Stethoscope-Doppler stethoscopes are used to  
  measure the Doppler effect of ultrasound waves 
  reflected from the various parts of the body, like 
  the beating of the heart. It detects motion by 
  detecting the change in frequencies.\\


6.3D-printed - A 3D-printed stethoscope is an open-
  source medical device meant for auscultation and 
  manufactured via means of 3D printing.The need for a 
  3D-stethoscope was borne out of a lack of 
  stethoscopes and other vital medical equipment 
  because of the blockade of the Gaza Strip.\\


7.Esophageal-Esophageal stethoscope is an instrument 
  for measuring and monitoring heart sound or 
  respiratory sound during operation. This device 
  consists of a thin, flexible, blind-ended tube 
  attached to a regular stethoscope usually by a small 
  plastic adaptor.\\

\clearpage

\section{Cheast X-Ray}
A chest X-ray is an imaging test that uses X-rays to look at the structures and organs in your chest. It can help your healthcare provider see how well your lungs and heart are working. Certain heart problems can cause changes in your lungs. Certain diseases can cause changes in the structure of the heart or lungs.\\

Chest X-rays produce images of your heart, lungs, blood vessels, airways, and the bones of your chest and spine. Chest X-rays can also reveal fluid in or around your lungs or air surrounding a lung.

\subsection{Why cheast X-Ray is done?}
A chest X-ray can reveal many things inside your body, including:\\

1.The condition of your lungs. Chest X-rays can detect cancer, infection or air collecting in the space around a lung, which can cause the lung to collapse. They can also show chronic lung conditions, such as emphysema or cystic fibrosis, as well as complications related to these conditions.\\

2.Heart-related lung problems. Chest X-rays can show changes or problems in your lungs that stem from heart problems. For instance, fluid in your lungs can be a result of congestive heart failure.\\

3.The size and outline of your heart. Changes in the size and shape of your heart may indicate heart failure, fluid around the heart or heart valve problems.
Blood vessels. Because the outlines of the large vessels near your heart — the aorta and pulmonary arteries and veins — are visible on X-rays, they may reveal aortic aneurysms, other blood vessel problems or congenital heart disease.\\

4.Calcium deposits. Chest X-rays can detect the presence of calcium in your heart or blood vessels. Its presence may indicate fats and other substances in your vessels, damage to your heart valves, coronary arteries, heart muscle or the protective sac that surrounds the heart. Calcified nodules in your lungs are most often from an old, resolved infection.\\

5.Fractures. Rib or spine fractures or other problems with bone may be seen on a chest X-ray.
Postoperative changes. Chest X-rays are useful for monitoring your recovery after you've had surgery in your chest, such as on your heart, lungs or esophagus. Your doctor can look at any lines or tubes that were placed during surgery to check for air leaks and areas of fluid or air buildup.\\

6.A pacemaker, defibrillator or catheter. Pacemakers and defibrillators have wires attached to your heart to help control your heart rate and rhythm. Catheters are small tubes used to deliver medications or for dialysis. A chest X-ray usually is taken after placement of such medical devices to make sure everything is positioned correctly.

\subsection{Importance of X-Ray in covid-19}
Like other pneumonias, covid-19 pneumonia causes the density of the lungs to increase. This may be seen as whiteness in the lungs on radiography which, depending on the severity of the pneumonia, obscures the lung markings that are normally seen; however, this may be delayed in appearing or absent.
\begin{figure}[h]
\centering
\includegraphics[scale=0.1]{cheast.jpg}
\caption{Cheast X-Ray}
\end{figure}
\clearpage

\section{Ear Thermometer}
A tympanic thermometer, or ear thermometer, is a hand-held device that measures the temperature of the eardrum using an infrared sensor. Tympanic thermometers are available at most grocery, drug, and medical supply stores.\\
Remote ear thermometers, also called tympanic thermometers, use an infrared ray to measure the temperature inside the ear canal. The pros: When positioned properly, infrared ear thermometers are quick and generally comfortable for children and adults.\\
\subsection{Pros and Cons}
Remote ear thermometers, also called tympanic thermometers, use an infrared ray to measure the temperature inside the ear canal.

The pros:

1.When positioned properly, infrared ear thermometers are quick and generally comfortable for children and adults.\

2.Infrared ear thermometers are appropriate for infants older than age 6 months, older children and adults.\

The cons:

1.Infrared ear thermometers aren't recommended for newborns.\

2.Earwax or a small, curved ear canal can interfere with the accuracy of a temperature taken with an infrared ear thermometer.\

\subsection{How accurate are Ear thermometers}
Tympanic thermometers, or digital ear thermometers, use an infrared sensor to measure the temperature inside the ear canal and can give results within seconds. If a person uses it correctly, the results will be accurate. However, ear thermometers may not be as accurate as contact ones.

\subsection{How to use?}
To use an ear thermometer, follow these steps:

1.Pull the top of the earlobe up and back.
Gently insert the tip of the thermometer into the ear canal toward the eardrum.\

2. The sensor should be pointing down the ear canal and not at the wall of the ear.\

3.Once the thermometer is in position, turn it on and wait for it to signal that the reading is complete.\

4.Remove the thermometer and read the temperature.
It is important to use a clean probe tip each time and to follow the manufacturer’s instructions.

\begin{figure}[h]
\centering
\includegraphics[scale=0.2]{ear.jpg}
\caption{Ear Thermometer}
\end{figure}
\clearpage

\section{PaceMaker}
A pacemaker is a small device that's placed (implanted) in the chest to help control the heartbeat. It's used to prevent the heart from beating too slowly. Implanting a pacemaker in the chest requires a surgical procedure. A pacemaker is also called a cardiac pacing device.\

\subsection{Types}
Types
Depending on your condition, you might have one of the following types of pacemakers.

1.Single chamber pacemaker. This type usually carries electrical impulses to the right ventricle of your heart.\

2.Dual chamber pacemaker. This type carries electrical impulses to the right ventricle and the right atrium of your heart to help control the timing of contractions between the two chambers.\

3.Biventricular pacemaker. Biventricular pacing, also called cardiac resynchronization therapy, is for people who have heart failure and heartbeat problems. This type of pacemaker stimulates both of the lower heart chambers (the right and left ventricles) to make the heart beat more efficiently.\


\subsection{What pacemaker do?}
Pacemakers work only when needed. If your heartbeat is too slow (bradycardia), the pacemaker sends electrical signals to your heart to correct the beat.

Some newer pacemakers also have sensors that detect body motion or breathing rate and signal the devices to increase heart rate during exercise, as needed.

A pacemaker has two parts:

1.Pulse generator. This small metal container houses a battery and the electrical circuitry that controls the rate of electrical pulses sent to the heart.\

2.Leads (electrodes). One to three flexible, insulated wires are each placed in one or more chambers of the heart and deliver the electrical pulses to adjust the heart rate. However, some newer pacemakers don't require leads. These devices, called leadless pacemakers, are implanted directly into the heart muscle.

\subsection{Symptoms of needing a pacemaker}\

1.Frequent fainting.\

2.Inexplicable fatigue (you get enough sleep and stay healthy, yet always feel tired).\

3.Inability to exercise, even lightly, without getting very winded.\

4.Frequent dizziness or lightheadedness.\

5.Heart palpitations or sudden, intense pounding in your chest (without exercise)\

\subsection{Mechanism of Pacemaker}
our heart’s sinus node is your natural pacemaker (found in the upper right chamber of the heart, known as the atrium). It sends an electrical impulse to make your heart beat. The job of a pacemaker is to artificially take over the role of your sinus node if it's not working properly.

Electrical impulses are sent by the pacemaker device to tell your heart to contract and produce a heartbeat. Most pacemakers work just when they’re needed – on demand. Some pacemakers send out impulses all of the time. Some pacemakers send out impulses all of the time, which is called fixed rate.
\begin{figure}[h]
\centering
\includegraphics[scale=0.6]{workpace.jpg}
\caption{Pacemaker}
\end{figure}
\clearpage

\section{Ultrasonic lithotripter}
Lithotripsy is a noninvasive procedure for treating kidney stones. Doctors use it with stones too large to pass through your urinary tract. This treatment method sends focused ultrasonic energy (shock waves) directly to the stone. Before beginning the treatment, your provider locates the stone using a special X-ray (fluoroscopy) or ultrasound.\

Types of lithotripsy-\

Ultrasonic lithotripsy. ...\

Electrohydraulic lithotripsy (EHL) ...\

Extracorporeal shock wave lithotripsy (ESWL)\


Ultrasonic lithotripsy is a safe and efficient procedure, and achieves a controlled destruction of the stone and allows a quick removal of the resulting fragments under endoscopic control.\

The principle of ultrasonic lithotripsy consists of transforming the electric energy into ultrasonic energy.\

Ultrasonic lithotripsy uses high frequency sound waves delivered through an electronic probe inserted into the ureter to break up the kidney stone. The fragments are passed by the patient or removed surgically.\

\subsection{How does Ultrasonic Lithotripter Works?}
Lithotripsy treats kidney stones by sending focused ultrasonic energy or shock waves directly to the stone first located with fluoroscopy (a type of X-ray “movie”) or ultrasound (high frequency sound waves). The shock waves break a large stone into smaller stones that will pass through the urinary system.

\subsection{Why to use Ultrasonic Lithotripter what is its advantage}
The major advantage of ultrasonic lithotripsy is the efficient combination of stone fragmentation and simultaneous fragment removal. Fragments smaller than 2 mm are aspirated through the hollow lithotrite along with the irrigation fluid. Larger fragments may be removed with forceps or baskets.

\subsection{How successful is extracorporeal shock wave lithotripsy?}
In those patients who are thought to be good candidates for this treatment, about 70 to 90 percent are found to be free of stones within three months of treatment. The highest success rates seem to be in those patients with mobile stones that are located in the upper portions of the urinary tract (kidney and upper ureter). After treatment, some patients may still have stone fragments that are too large to be passed. These can be treated again if symptoms persist

\begin{figure}[h]
\centering
\includegraphics[scale=0.7]{litho.jpg}
\caption{Ultrasonic lithotriper}
\end{figure}

The principle of ultrasonic lithotripsy consists of transforming the electric energy into ultrasonic energy. The electricity produced by a generator  is transmitted to the transducer, determining excitation of a piezoelectric crystal. The crystal vibrates at a specific frequency and generates an acoustic wave with a frequency of 23–25 kHz. High sounds with a metallic character of different tones can be perceived under operational frequencies, varying according to the aspirated liquid and to the pressure gradient exerted by the sonotrode on the stone. However, there is also a high level of noise, imperceptible to the human ear, which can reach levels of 98 dB (Segura and LeRoy, 1984).
\clearpage

\begin{figure}
\centering
\includegraphics[scale=3]{thankss.jpg}
\end{figure}
















\end{document}