\documentclass[12pt]{report}

\usepackage{hyperref}
\hypersetup{colorlinks=true,citecolor=blue,
linkcolor=blue,urlcolor=blue}

\title{\underline{ASSIGNMENT 5} \\
Emerging technology in Health care  }
\author{Roll.no-21111067\\Tushar Baghel\\tbtusharbaghel7@gmail.com\\BIOMEDICAL DEPARTMENT\\NIT RAIPUR\\}
\usepackage{graphicx}
\graphicspath{{images/}}

\begin{document}

\begin{figure}
\centering
\includegraphics[scale=0.5]{nitrr.jpg}
\end{figure}
\maketitle
\clearpage
\tableofcontents
\clearpage

\section{ACKNOWLEDGEMENT}\


I would like to express my special thanks of gratitude to my teacher Dr.Saurab Gupta sir who gave me the golden opportunity to this assignment, which also helped me in doing reasearch and gaining knowledge about future of healthcare and emerging technology.\paragraph{}

Secondly,I would also like to thank my parents and friends who helped me in suggesting information for this project.\paragraph{•}


Thanking You\paragraph{•}

Tushar Baghel 

\clearpage


\section{What is Heath Care System}

A health system, also known as health care system or healthcare system, is the organization of people, institutions, and resources that deliver health care services to meet the health needs of target populations.\\
The goal of a healthcare system is to enhance the health of the population in the most effective manner possible in light of a society's available resources and competing needs.

\section{Present Heatlh Care}
The doctor-to-population ratio in India is 1:214 Though hospitals, dispensaries, public health centers and other medical facilities are present, they are not sufficient to cater to the growing needs of India's substantial population. Rural access to quality medical service has to be improved.\\
The inadequate manpower of doctors in public sector hospitals is also a concern for health authorities. Furthermore, the infrastructure required in the hospitals, like medicine, furniture and equipment, are not adequate to serve the population. Compounding the problem, government spending on healthcare services is not up to the World Health Organization (WHO) norms of gross domestic product in healthcare.\\

\section{why Technology is needed?}

Innovations and Technology in healthcare can drive economic growth by improving efficiency and increasing productivity, as well as optimizing patient outcomes.Thus inovation are much needed today so that in future our Health Care goes better and can help mankind.

\clearpage


\section{Top Inovation in present Years and Emerging Technology.}\paragraph{1.}
Artificial Intelligence (AI)-Artificial intelligence (AI) is the ability of a computer or a robot controlled by a computer to do tasks that are usually done by humans because they require human intelligence and discernment.\\
What are the 4 types of AI?\\
According to the current system of classification, there are four primary AI types: reactive, limited memory, theory of mind, and self-aware.\\
With the potential to radically transform healthcare, artificial intelligence can help professionals make better judgments and reduce human error and the risk of preventable scenarios. From radiology tools and immunotherapy for cancer patients to identifying infectious disease patterns, advanced technology helps develop more efficient and precise interventions. As learning algorithms evolve and become more accurate, they are likely to significantly impact healthcare services, including diagnostic approaches, treatments, and care processes.  \\
“A year spent in artificial intelligence is enough to make one believe in God.”\\
Example are-  Maps and Navigation,Digital Assistants Social Media,E-Payments etc.\paragraph{2.}
3D printing, or additive manufacturing, is the construction of a three-dimensional object from a CAD model or a digital 3D model. The term "3D printing" can refer to a variety of processes in which material is deposited, joined or solidified under computer control to create a three-dimensional objec.\\3D printing is used for the development of new surgical cutting and drill guides, prosthetics as well as the creation of patient-specific replicas of bones, organs, and blood vessels.\paragraph{3.}
Robotics-Medical robots assist with surgery, streamline hospital logistics, and enable providers to give more direct attention to patients. Robots in the medical field are transforming how surgeries are performed, streamlining supply delivery and disinfection, and freeing up time for providers to engage with patients.\\
What is the Future of Robotics in Healthcare?\\
With new applications and features, healthcare robots are expected to increase the quality, operational efficiencies, accuracy, and safety in healthcare service delivery. The advancement in AI will provide a new dimension to robotics. As expected the combination of Artificial intelligence and robotics will make the operation faster and much safer.\paragraph{4.}
 

DIGITAL THERAPEUTICS\\
Patients that have chronic illnesses often require ongoing care from their physicians. This care can include patient education, symptom monitoring, medication adjustment, and behavioral changes. Not only is this care costly, but it is also very time-consuming for both medical staff and patients. Now, there are new digital therapeutics that can fill this role.

Digital therapeutics are prescribed by a doctor to a patient for their particular medical condition.\paragraph{5.}

Blockchain technology in healthcare\\
In healthcare, Blockchain has a wide range of applications and functions. The ledger technology helps healthcare researchers uncover genetic code by facilitating the secure transfer of patient medical records, managing the drug supply chain, and facilitating the safe transfer of patient medical records.\\

How blockchain technology can transform the healthcare sector?\\
Firstly, it allows private information to be securely shared without the need to copy it, which can help reduce mistakes in healthcare records. The data is also time-stamped, which gives it greater security. Blockchain has the ability to securely streamline payments across healthcare environments.\paragraph{6.}

Wearable technology in healthcare\\
Wearables are small electronic devices that, when placed on your body, can help measure temperature, blood pressure, blood oxygen, breathing rate, sound, GPS location, elevation, physical movement, changes in direction, and the electrical activity of the heart, muscles, brain, and skin.\\
How does wearable technology work in healthcare?\\
In healthcare, the Wearable IoT (WIoT) is a network of patient-worn smart devices (e.g., electronic skin patches, ECG monitors, etc.), with sensors, actuators and software connected to the cloud that enable collection, analysis and transmitting of personal health data in real time.\\
Common examples of wearable technology include: Smart jewelry, such as rings, wristbands, watches and pins.\paragraph{7.}
QUANTUM COMPUTING\\
In the healthcare industry, quantum computing could enable a range of disruptive use cases for providers and health plans by accelerating diagnoses, personalizing medicine, and optimizing pricing. Quantum-enhanced machine learning algorithms are particularly relevant to the sector.\\
We are still in the early stages of quantum computing, but this technology has already been used in combination with machine learning to quickly recognize medical tools and annotations during cataract surgery—the most commonly performed surgery in the world. While much of quantum computing’s potential is still to be actualized, past performance has already caused significant optimism for applications in medical imaging, genomics, and drug discovery.




\section{Conclusion}
In present there are many problem which can be solved by technolgy and Inovation in future thus technology will play a major role in upcoming future healthcare. Technology such as artificial intelligence and robotics are some technology which will be very beneficial in upcoing future healthcare.Future healthcare will be focousing on that area where present heathcare is lagging behind and this will help to improve our heathcare system and make it effecient.This Emerging Technology such as blockchain and quantum computing can and will contribute in healthcare.





\clearpage

\begin{figure}
\centering
\includegraphics[scale=03]{thankss.jpg}
\end{figure}











\clearpage







\end{document}