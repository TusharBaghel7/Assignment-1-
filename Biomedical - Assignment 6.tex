\documentclass[12pt]{report}

\usepackage{hyperref}
\hypersetup{colorlinks=true,citecolor=blue,
linkcolor=blue,urlcolor=blue}

\title{\underline{ASSIGNMENT 6} \\
5 solution to covid 19 provided by biomedical engineers  }
\author{Roll.no-21111067\\Tushar Baghel\\tbtusharbaghel7@gmail.com\\BIOMEDICAL DEPARTMENT\\NIT RAIPUR\\}
\usepackage{graphicx}
\graphicspath{{images/}}

\begin{document}

\begin{figure}
\centering
\includegraphics[scale=0.5]{nitrr.jpg}
\end{figure}
\maketitle
\clearpage
\tableofcontents
\clearpage

\section{ACKNOWLEDGEMENT}\


I would like to express my special thanks of gratitude to my teacher Dr.Saurab Gupta sir who gave me the golden opportunity to this assignment, which also helped me in doing reasearch and gaining knowledge about covid 19 and its solution .\paragraph{}

Secondly,I would also like to thank my parents and friends who helped me in suggesting information for this project.\paragraph{•}


Thanking You\paragraph{•}

Tushar Baghel 

\clearpage

\section{What do Biomedical Engineers do?}
The role of a Biomedical Engineer includes designing biomedical equipment and devices to aid the recovery or improve the health of individuals. This can include internal devices, such as stents or artificial organs, or external devices, such as braces and supports (orthotics). It can also include creating and adapting medical equipment. It’s a role that requires excellent knowledge of computing, biology and engineering, an inventive nature, and good problem solving skills.
In this pandemic covid-19 biomedical engineers helped doctors and patient.

\section{5 solution to covid 19 provided by biomedical engineers}
5 solution provided by biomedical engineers are-\\
1.Ventilators\\
2. Continuous Positive Airway Pressure (CPAP).\\
3. Patient Moniters.\\
4.Oximeter\\
5.Oxygen Concentrator.\\




\subsection{Ventilators}
Patients who cannot breathe spontaneously need to be put on a ventilator. Ventilators are capable of replacing the breath function and patients in an advanced state of respiratory distress are usually intubated and sedated at the beginning of the treatment.

Ventilators are capable of replacing the breath function and patients in an advanced state of respiratory distress are usually intubated and sedated at the beginning of the treatment. They are complex systems providing the healthcare professionals with a lot of flexibility to adapt the assisted breathing settings and to be able to wean recovering patients off the ventilator gradually.

\subsection{Continuous Positive Airway Pressure (CPAP)}
The next step up in treating COVID-19 patients is Continuous Positive Airway Pressure (CPAP) which is initially intended to prevent airways collapse in sleep apnoea patients, but has been shown to be beneficial to COVID patients if applied early enough in the progression of the disease.\\

Continuous positive airway pressure therapy (CPAP) uses a machine to help a person who has obstructive sleep apnea (OSA) breathe more easily during sleep. A CPAP machine increases air pressure in your throat so that your airway doesn't collapse when you breathe in.

\subsection{Patient monitoring}
During the ongoing COVID-19 pandemic, Internet of Things- (IoT-) based health monitoring systems are potentially immensely beneficial for COVID-19 patients. This study presents an IoT-based system that is a real-time health monitoring system utilizing the measured values of body temperature, pulse rate, and oxygen saturation of the patients, which are the most important measurements required for critical care. This system has a liquid crystal display (LCD) that shows the measured temperature, pulse rate, and oxygen saturation level and can be easily synchronized with a mobile application for instant access.

\subsection{Oximeter}
A pulse oximeter is a device that checks to see how much oxygen your blood is carrying. It's a fast, simple way to learn this information without using a needle to take a blood sample.

Usually a small clip is put on the end of your finger. (Sometimes it's put on your toe or earlobe.) The device shines a light beam through the skin. It estimates your oxygen level by measuring the percentage of your blood that's carrying oxygen. Your oxygen level (or oxygen saturation, SpO2) shows on the display screen.

These oximeter were very helpful in covid 19 people used it from thier homes to check thier oxygen level helped them to recognise symptoms of covid 19.

\subsection{Oxygen Concentrator}
An oxygen concentrator uses the air to filter oxygen and is the best solution to oxygen supply at home. Patients access this oxygen through a mask or a cannula. It is generally used for patients with respiratory issues and with the ongoing crisis of COVID-19, can be very useful for patients whose oxygen levels drop.

“A concentrator is a device that can provide oxygen for hours, it does not need to be replaced or refilled with anything. However, to help people with oxygen supplementation, one needs to know the right way to use an oxygen concentrator,” said Dr Bela Sharma, additional director, Internal Medicine, Fortis Memorial Research Institute, Gurugram.





\clearpage

\begin{figure}
\centering
\includegraphics[scale=03]{thankss.jpg}
\end{figure}











\clearpage







\end{document}